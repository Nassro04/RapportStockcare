\chapter{Contexte et Étude Technique}

\section{Introduction}
La gestion efficace des stocks pharmaceutiques est cruciale pour garantir la disponibilité des médicaments et la sécurité des patients. Le projet Stockcare vise à répondre aux défis courants tels que la rupture de stock, le gaspillage dû à la péremption et le manque de traçabilité.

\section{Objectifs du Projet}
\begin{itemize}
    \item Suivi en temps réel des niveaux de stock.
    \item Gestion des dates de péremption (FEFO - First Expired, First Out).
    \item Traçabilité des entrées et sorties par utilisateur et lot.
    \item Tableaux de bord décisionnels.
\end{itemize}

\section{Choix Technologiques}

\subsection{Backend : Spring Boot}
Nous avons choisi Spring Boot pour sa robustesse et sa capacité à gérer des architectures complexes. Il facilite la création d'APIs REST sécurisées et performantes.

\subsection{Frontend : Next.js (React)}
Next.js offre une excellente expérience développeur et utilisateur, avec le rendu côté serveur (SSR) et une gestion optimisée des routes, essentiel pour une application dashboard réactive.

\subsection{Base de Données : PostgreSQL}
Pour la persistance des données, PostgreSQL a été retenu pour sa fiabilité et sa conformité aux standards SQL, crucial pour des données critiques comme les stocks médicaux.

\subsection{Conteneurisation : Docker}
L'ensemble de l'application est conteneurisé pour faciliter le déploiement et garantir la cohérence entre les environnements de développement et de production.
