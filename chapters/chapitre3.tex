\chapter{Conception et Architecture}

\section{Architecture Globale}
L'application suit une architecture moderne séparant le frontend du backend, communiquant via une API REST.

\begin{figure}[H]
    \centering
    % Placeholder pour le diagramme d'architecture
    \fbox{\begin{minipage}{0.8\textwidth}
        \centering
        \vspace{2cm}
        \textbf{[Insérer ici le diagramme d'architecture globale]}
        \vspace{2cm}
    \end{minipage}}
    \caption{Architecture Client-Serveur de Stockcare}
    \label{fig:archi_globale}
\end{figure}

\section{Diagrammes UML}

\subsection{Diagramme de Classe}
Le diagramme de classe ci-dessous illustre les relations entre les entités métier (Produits, Fournisseurs, Commandes).

\begin{figure}[H]
    \centering
    % Placeholder pour le diagramme de classe
    \fbox{\begin{minipage}{0.9\textwidth}
        \centering
        \vspace{4cm}
        \textbf{[Insérer ici le Diagramme de Classe]}
        \vspace{4cm}
    \end{minipage}}
    \caption{Diagramme de Classe}
    \label{fig:class_diagram}
\end{figure}

\subsection{Diagramme de Séquence : Sortie de Stock}
Ce diagramme détaille le flux d'interaction lors de la sortie d'un produit du stock par un pharmacien.

\begin{figure}[H]
    \centering
    % Placeholder pour le diagramme de séquence
    \fbox{\begin{minipage}{0.9\textwidth}
        \centering
        \vspace{4cm}
        \textbf{[Insérer ici le Diagramme de Séquence]}
        \vspace{4cm}
    \end{minipage}}
    \caption{Diagramme de Séquence - Sortie de Stock}
    \label{fig:seq_diagram}
\end{figure}
